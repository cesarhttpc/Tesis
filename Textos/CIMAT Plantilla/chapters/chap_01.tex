\chapter{Introducción}

El estudio de la violencia en México no es un tema nuevo, sino de antaño para la sociedad, así como la preocupación por resolverlo. En consecuencia, a través del tiempo se han planteado diversas explicaciones, tanto teóricas como empíricas para abordar el fenómeno e identificar factores que llevan a los grupos humanos a ejercer la violencia en los territorios. Estas explicaciones han sido abordadas desde diversas visiones como  las geográficas, políticas, económicas, y sociológicas, por nombrar las más estudiadas. Ante una realidad tan compleja como es el estudio de la violencia, observamos la necesidad de abordarlo desde las ciencias estadísticas y computacionales, y es desde esta necesidad que desarrollamos esta tesina \ldots


\section{Problema} \label{sec:problem}

México actualmente atraviesa una crisis de violencia e inseguridad. El aumento de la violencia se debe, en gran medida, a la estrategia impulsada por el gobierno en el año 2006 \citep{Atuesta2017}.



\section{Contribución} \label{sec:contribution}

En literatura existe una basta colección de trabajos que abordan el problema de la violencia el México por el crimen organizado analizando sus causas y efectos. Algunos de éstos estudian las causas, y se enfocan en los tipos de actividades en las que operan las organizaciones criminales, como la extorsión, el tráfico de drogas, secuestros, etcétera. En otros trabajos se analizan los efectos, y que entre ellos puede ser la violencia contra grupos específicos, como lo son el sector periodístico y funcionarios políticos \citep{Rios2012tendencias}. Otros como \citet{Youngers2014} hablan sobre los esfuerzos y cambios realizados por los gobiernos de Latinoamérica en cuanto a políticas de drogas, como una alternativa al combate directo contra el narco.

\section{Organización de la tesis}






