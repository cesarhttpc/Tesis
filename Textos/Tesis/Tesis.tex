\documentclass[letterpaper,12pt,twoside]{book}  % book class
\usepackage[spanish,es-nodecimaldot,es-tabla]{babel} % language/hyphenation
\usepackage[utf8]{inputenc}  % input encoding
\usepackage{graphicx} % graphics library (REQUIRED by CIMATpreamble)
\usepackage[svgnames,table]{xcolor}  % color library (REQUIRED by CIMATpreamble)
\usepackage{CIMATpreamble}  % definition of the CIMAT cover page
\usepackage{mypreamble} % custom preamble
\graphicspath{{img/}}  % set path for images folder

%-----------------------------------------------------------------------------%
% Coverpage template requested by CIMAT for the master thesis in LaTeX version 

%-----------------------------------------------------------------------------%
% README: Instructions & Comments
%
% > The package 'CIMATpreamble' is required for the CIMAT cover page and 
%   it depends on the declared libraries above
% > To use the COVER template only fill the Cover/Titlepage block 
%   with your information and make sure the '\maketitle' command is enabled
%
% > The rest of the LaTeX code was built according to the guideline provided by 
%   the ShareLatex/Overleaf team:
%   "How to Write a Thesis in LaTeX (Part 1): Basic Structure"; you will find  
%   the custom settings (link colors, caption specs., etc.) in the 
%   'mypreamble.sty' file, so you are free to use or modify that code 
% > Much of the code for the cover page is based on the Overleaf template    
%   created by Salvador Pedraza Espitia: 
%   https://www.overleaf.com/latex/templates/unam-dissert/mypjkyrmhgns
% > The essential idea of this code is to provide a minimal and functional  
%   LaTeX document, allowing us to focus on the content of our thesis work;
%   I hope it be useful for you :) !
% 
% > Credits: Kenny Yahir Mendez Ramirez, Cristal E. Mendez Ramirez
% > For comments and suggestions please send a e-mail to kenny.yahir.mz@gmail.com

%-----------------------------------------------------------------------------%
% Cover/Title page: fill out with your personal information

\author{César Isaí García Cornejo}
\documentType{T E S I S}  % change to "T E S I N A" if it is the case
\title{Cuantificación de incertidumbre bayesiana aproximada en problemas inversos de ODE}
\degree{Maestro en Probabilidad y Estadística}
\supervisor{Dr. José Andrés Christen Gracia}
% \supervisorSecond{Dra. M. F.}  % optional command
\cityandyear{Guanajuato, Guanajuato, Verano 2024}

%-----------------------------------------------------------------------------%
% Document body

\begin{document}

% Coverpage
\maketitle  % this enable the CIMAT title page!

\thispagestyle{empty}  % blank page after title page

\frontmatter

% Dedication
\chapter*{}
\begin{flushright}%
  \emph{A mi madre \textbf{Rosa Cornejo} quien me apoyo emocionalmente y me incentivo a conseguir mis metas. \\A mi padre \textbf{Julio García} quien con su esfuerzo logro darme la oportunidad de perseguir la profesionalización.\\
  A mi novia \textbf{Zaira Martínez} quien me dío amor y apoyo durante la elaboración de la tesis así como en la maestría. }
  \thispagestyle{empty}
\end{flushright}

% Abstract
% \chapter*{Resumen}
% \addcontentsline{toc}{chapter}{Resumen}

% Estudio del problema inverso para modelos descritos por ODEs con forward map aproximado.
% \\
% % Keywords
% \paragraph{Palabras clave:} Cuantificación de la incertidumbre, problemas inversos, inferencia bayesiana, ecuaciones diferenciales, MCMC

% Acknowledgements
\chapter*{Agradecimientos}
\addcontentsline{toc}{chapter}{Agradecimientos}

A mis padres que me apoyaron incondicionalmente confiando en mí dándome el impulso que necesitaba para concluir la maestría antes y durante el periodo de la misma. Agradecer a mi asesor Andrés Christen por su valioso apoyo como asesor, quien además se mostró disponible y generoso. Agradecer a mi novia Zaira Martínez por su apoyo, comprensión y disciplina que nos sacó adelante día con día. A mis compañeros y ahora amigos cuyo apoyo mutuo fue central en nuestro crecimiento denotando sus fuertes valores como el compañerismo, empatía y amistad.

% Table of contents and list of figures
\tableofcontents

\listoffigures

% Chapters
\mainmatter

\spacing{1.5}  % interline  (double) space

% put the content of your chapters in the .tex files located in the 'chapters/' folder; 
% the use of '\include' instead of '\input' command could enhance the  compilation process, see the reference below

\chapter{Introducción}

El estudio de la violencia en México no es un tema nuevo, sino de antaño para la sociedad, así como la preocupación por resolverlo. En consecuencia, a través del tiempo se han planteado diversas explicaciones, tanto teóricas como empíricas para abordar el fenómeno e identificar factores que llevan a los grupos humanos a ejercer la violencia en los territorios. Estas explicaciones han sido abordadas desde diversas visiones como  las geográficas, políticas, económicas, y sociológicas, por nombrar las más estudiadas. Ante una realidad tan compleja como es el estudio de la violencia, observamos la necesidad de abordarlo desde las ciencias estadísticas y computacionales, y es desde esta necesidad que desarrollamos esta tesina \ldots


\section{Problema} \label{sec:problem}

México actualmente atr \cite{Kahle2013ggmap}




\section{Contribución} \label{sec:contribution}

En literatura
\section{Organización de la tesis}









\chapter{Antecedentes}

En el capítulo precedente relata la estructura subyacente de la modelación. Formalizaremos y detallaremos los lineamientos  del sector referente al problema inverso, cuya gestión no es única para cada caso, pues esta sujeto a las vicisitudes del modelo en cuestión. A diferencia de la mayoría de los enfoques al problema inverso, el enfoque bayesiano goza de ser un método apropiado pues para los modelos contemplados aquí la metodología es análoga (\cite{tarantola2005inverse}.)

% Se describe el enfoque bayesiano del problema inverso, que es un paradigma ampliamente estudiado, además de contrastar este enfoque con 

% Otro enfoque al problema inverso es el operativo, el cual se aborde desde el problema directo como un operador actuando en el modelo, y el problema inverso como un operador inverso de este. Este último enfoque sale de los límites de estudio del texto. 

\section{Problema inverso}

El estudio del problema inverso se remonta a mediados del siglo 17. Los físicos contemporáneos interesados en la relación causa-efecto de diferentes fenómenos físicos lograron predecir el comportamiento en la dinámica intrínseca. Así, para la ley de gravitacional de newton, una vez obtenidas las masas de los cuerpos celestes (causas) es realizable predecir las trayectorias que siguen los cuerpos en el tiempo (efecto). Estudiar los efectos de un fenómeno dada ciertas causas es el problema directo perteneciente al sector de modelación directa. El problema inverso toma la dirección opuesta, dados los efectos, es de interés investigar las causas que lo ocasionaron. Retomando un ejemplo newtoniano, dado un campo gravitatorio circundando cierto cuerpo celeste, es de interés conocer la masa de dicho cuerpo. Enfaticemos que el problema inverso adolece de no ser único, suele ser el caso que a ciertos efectos se correspondas varias causas. el hecho de que no exista una mapeo uno a uno entre causa y efecto complica el problema inverso. Sin embargo, la información que se tenga a priori del fenómeno es crucial para determinar con una menor ambigüedad las causas (\cite{tarantola2005inverse}). 

\subsection*{Modelos descritos por ODEs}

Para describir fenómenos los modelos matemáticos pueden ser de una suntuosa cantidad de clases o estilos. Con fines pedagógicos, considere que se desea describir los costos de la renta $Y$ de inmuebles en cierta ciudad (efecto). Supongase que de análisis previos se concluyo que dicho fenómeno esta vinculado al tamaño del inmueble $X_1$, número de habitaciones $X_2$, ingreso medio por habitante $X_3$ (causas). Un modelo plausible es por modelos lineales generalizados
\begin{align}
    Y = G(\beta_0 + \beta_1 X_1 + \beta_2 X_2 + \beta_3 X_3) + \varepsilon
    \label{2.1.1.01}
\end{align}
donde $G$ es la función liga, $\varepsilon \sim N(0,\sigma^2)$ un error aleatorio y $\beta = (\beta_0, \beta_1, \beta_2, \beta_3)$ son los parámetros del modelo (\cite{dobson2018introduction}). El problema directo considera predecir el valor de $Y$ conociendo $X = (X_1,X_2,X_3)$ y los parámetros del modelo $\beta$. Para esta clase de modelos el problema directo no representa complejidad. En contraste, el problema inverso precisa estimaciones de $\beta$  y $\sigma$ dada muestras de $Y$ y $X$, siendo un problema no trivial. 

% Introducir los modelos por EDOs
El análisis del problema inverso para modelos en general es bastante amplio. Por ello es necesario restringirlo a una clase menor de modelos, siendo estos los modelos dados por ODEs. Considerese un modelo para describir al dinámica de $y(t)$ a lo largo del tiempo $t$. Los modelos considerados son aquellos que se pueden expresar de la forma
\begin{align*}
    G(t,y(t),y'(t),y''(t),...) = 0
\end{align*}
que es una ODE con $G$ una función continua conocida.

Más aún, se puede obtener el análisis del problema inverso para sistemas de ecuaciones diferenciales ordinarias. Esto es
\begin{align*}
    \left\{
        \begin{matrix}
        G_1(t,y_1(t),y_1'(t),...,y_2(t),y_2'(t),..., y_d(t),y_d'(t),...)=0\\
        G_2(t,y_1(t),y_1'(t),...,y_2(t),y_2'(t),..., y_d(t),y_d'(t),...)=0\\
        \vdots
        \\ 
        G_d(t,y_1(t),y_1'(t),...,y_2(t),y_2'(t),..., y_d(t),y_d'(t),...)=0\\
       \end{matrix}
    \right.
\end{align*}
con $G_1, G_2, ..., G_d$ funciones continuas conocidas (\cite{apostol2019calculus}).

Observemos que la restricción implica restringir las variables en un soporte continuo, por lo que modelos con variables discretas deben de modificarse. En el capítulo 3 se detallan tres modelos de los cuales se consideraron los análisis del problema inverso. Dentro de estos se consideran modelos como la dinámica de caída libre sujeto a fricción, donde la distancia recorrida en caída se describe por
\begin{align}
    mx''(t) = mg - bx'(t)
    \label{2.1.1.03}
\end{align}
con $m, g, b$ parámetros del modelo.

Por el teorema de existencia y unicidad en ecuaciones diferenciales; dadas las condiciones iniciales, los parámetros del modelo definen unívocamente el modelo (\cite{kelley2010theory}). Consideremos el espacio de parámetros $\Theta$ que a su vez es la familia de \textit{submodelos} posibles descrito por la ODE. 


El problema directo es entonces aquel que dado un $\theta \in \Theta$ obtiene la trayectoria o solución de la ecuación diferencial. Dicho problema esta bien definido, pues resolver la ODE es posible al menos numéricamente. Denotemos por $\mathcal{Y}$ a las soluciones posibles de la ecuación diferencial. De esta forma existe un mapeo del espacio $\Theta$ a $\mathcal{Y}$ el cual llamaremos \textbf{forward map}. Así, el forward map es 
\begin{align*}
    \theta \mapsto F(\theta) 
\end{align*}
donde $\theta \in \Theta$ y $F(\theta) \in \mathcal{Y}$.

Para el modelo dado en (\ref{2.1.1.03}) una vez dado $\theta = (m,g,b)$ la solución $x(t) \in \mathcal{Y}$ se obtiene del forward map, que en otras palabras es simplemente la solución analítica o numéricamente de la ecuación diferencial.

\section{Solución bayesiana a  problemas inversos}


Principalmente existen dos razones para que las observaciones de un modelo no concuerden exactamente con las predicciones dadas por el mismo. La primera se debe a errores de medición por la incertidumbre de los aparatos de medición. El segundo motivo se debe a los defectos propios del modelo, pues a pesar de que el modelo pretende describir el fenómeno de interés estos nunca son lo mismo. Así se considera la vaga interpretación que jacta a los modelos como aproximaciones de los fenómenos. La relevancia de los errores abre una puerta a la cuantificación de la incertidumbre.

Por otro lado, el paradigma bayesiano para problemas inversos se centra en cuantificar la incertidumbre en los parámetros del modelo.Es decir, su objetivo es establecer una medida de probabilidad posterior a las observaciones de la trayectoria del modelo. Equivalentemente, se busca la distribución de probabilidad $\pi(\theta|\mathbf{y})$ donde $\mathbf{y} = (y_1,...,y_n)$ son las observaciones de la trayectoria a lo largo del tiempo $t_1, t_2, \cdots, t_n$. Partiendo de la información acerca de los parámetros previo a cualquier observación, se propone una distribución de probabilidad $\pi(\theta)$ llamada distribución a priori, y tras aplicar el Teorema de Bayes obtener la distribución posterior (\cite{wasserman2013all}).

Como ya se ha mencionado, no se espera que las observaciones de la trayectoria coincidan con las predicciones del modelo. En virtud de lo mencionado, ajustamos los errores conforme a una distribución normal. Para modelos dinámicos, aquellos que evolucionan con el tiempo, la predicción se obtiene del forward map con un vector de parámetros $\theta \in \Theta$ dado.
De esta forma, obtenemos una función en el tiempo $y(t)$, Es decir, se tiene la igualdad $F(\theta) = y(t)$. Luego, se requieren las predicciones dadas por el modelo a tiempos fijos $t_i$, la cual simplemente es la evaluación $y(t_i)$, para todo $i \in \{1,...,n\}$. Usando la notación alterna con el forward map, se tiene $F_{\theta}(t_i)$ como la predicción a tiempo fijo.

De esta forma, se establece que las discordancias entre observaciones y predicciones siguen una distribución normal de la forma
\begin{align*}
    y_i = F_{\theta} (t_i) + \varepsilon_i, \:\:\:\:\:\: \varepsilon_i \sim N(0,\sigma^2).
\end{align*}
Dicho ajuste es elegido por antonomasia, debido a la predilección  dentro de los de su clase. Cualquier otra distribución o forma funcional para ajustar los errores entre medición y modelo es un tópico interesante, sin embargo sale del propósito de la tesis por lo que se restringe al estudio clásico de ajuste (\cite{berger2013statistical}).

En consecuencia, se ha establecido una distribución para las observaciones $y_i$ las cuales tienen asociada una verosimilitud sobre $\theta$ y $\sigma$. Como se consideran errores independientes, pues se tratan de errores de medición, la verosimilitud se sigue de
\begin{align*}
    \mathcal{L}(\theta,\sigma) &= f(\mathbf{y}|\theta) = \prod_{i = 1}^{n} \frac{1}{\sqrt{2\pi \sigma^2}} \exp \left \{ -\frac{1}{2\sigma^2}\left(y_i - F_{\theta}(t_i)\right)^2 \right \} , 
\end{align*}
simplificando
\begin{align}
    f(\mathbf{y}|\theta) = \left(\frac{1}{2\pi \sigma^2}\right) ^{n/2}\exp \left \{  -\frac{1}{2\sigma^2}\sum_{i = 1}^{n} \left(y_i - F_{\theta}(t_i)\right)^2 \right \},
    \label{2.2.03}
\end{align}
Del teorema de Bayes, se obtiene la distribución posterior para los parámetros
\begin{align}
    \pi(\theta| \mathbf{y})  = \frac{f(\mathbf{y}|\theta)\pi(\theta)}{\int f(\mathbf{y}|\theta)\pi(\theta)d \theta},
    \label{2.2.04}
\end{align}
la constante de integración $h(\mathbf{y}) = \int f(\mathbf{y}|\theta)\pi(\theta)d \theta$ es la constante de normalización para la distribución posterior. 

Salvo en análisis conjugado, donde la distribución a priori y la distribución posterior pertenecen a la misma familia, la obtención analítica de la constante de normalización $h(\mathbf{y})$ es un problema complejo. De aquí surge la necesidad de métodos numéricos para determinar la distribución posterior con precisión (\cite{robert1999monte}). 


Utilizar métodos numéricos de integración para $h(\mathbf{y})$ solamente es plausible para modelos de un solo parámetro $(d=1)$, es decir el espacio parametral $\Theta \subset \mathbb{R}$. En el caso de más de un parámetro $(d > 1)$ aún es posible usar métodos numéricos, sin embargo tienen un desempeño deficiente. 

A mediados del siglo XX, la estadística bayesiana inició un crecimiento sin precedentes tras descubrir e implementar métodos Montecarlo. Los métodos Montecarlo son un conjunto de algoritmos iterativos no deterministas con fin de estimar el valor de cierto calculo. El método Markov Chain Monte Carlo (MCMC) resulta ser conveniente para la estadística bayesiana, no solo porque evita el calculo de la constante de integración $h(\mathbf{y})$ sino por la generalidad que propicia al poder utilizarse para estimar una basta familia de distribuciones. Además, la estimación por este método tiene un error absoluto que decrece como $\frac{1}{N}$ en virtud del teorema del límite central (\cite{casella2024statistical}), que es de orden menor que aquellos dados por métodos numéricos de integración. La asiduidad del método MCMC, tras prevalecer en el tiempo, nos da una idea de lo apropiado que es para la estadística bayesiana.

\section{Método MCMC}

Obtener la distribución posterior (\ref{2.2.04}) con el método de Markov Chain Monte Carlo (MCMC) evita el calculo numérico de $h(\mathbf{y}$) como se ha mencionado previamente. En su lugar, MCMC estima la distribución posterior con una generosa muestra simulada de la misma distribución posterior salvo la constante de normalización. Por ende, con la misma muestra permite obtener estimaciones a cualquier momento de la distribución posterior asimismo la estimación de las distribuciones marginales. Los métodos MCMC busca encontrar una cadena de Markov $X^{(1)},X^{(2)},\cdots$ cuya distribución estacionaria sea la \textit{distribución objetivo} \footnote{En aplicaciones a la estadística bayesiana usando MCMC, la distribución posterior es la distribución objetivo.} $f(x)$ pese a que la misma no esté normalizada. 

\subsection*{Algoritmo Metropolis-Hastings}

El algoritmo de Metropolis-Hastings es un método de MCMC para generar muestras de una distribución objetivo $f(x)$ partiendo de muestras de una distribución propuesta $q(y|x)$ que no son necesariamente simétricas. Al igual que todo método MCMC, el algoritmo de Metropolis-Hastings genera una cadena $X_0, X_1, \cdots, X_N$ construyendo recursivamente la cadena dependiendo solamente del estado previo, es decir preservando la propiedad de Markov. El estado inicial $X_0$ se toma aleatoriamente. Luego, teniendo hasta el estado $X_i$, el estado $X_{i+1}$ se obtiene siguiendo 
\begin{enumerate}
    \item Generar una propuesta $Y \sim q(y|X_i)$.
    \item Calcular $\rho$ 
    \begin{align*}
        \rho = \min \left \{ \frac{f(y)q(x|y)}{f(x)q(y|x)}, 1  \right \} 
    \end{align*}
    \item Realizar un experimento Bernoulli con probabilidad de éxito $\rho$.
    \begin{align*}
        X_{i+1} = \left\{\begin{matrix}
            Y & \text{probabilidad} \rho  \\ 
            X_{i}& \text{probabilidad} 1- \rho  
           \end{matrix}\right.
    \end{align*}
\end{enumerate}

En el siguiente gráfico se muestra un diagrama de flujo del algoritmo Metropolis-Hastings.

\begin{figure}[H] 
    \centering 
    \includegraphics[width = 14 cm]{MCMC.png} 
    % \caption{}
    \label{Fig. M-H}
\end{figure} 

El algoritmo previamente esbozado a manera de opúsculo, subrepticiamente converge en distribución estacionaria a la distribución objetivo. Indudablemente dicha convergencia está asegurada por el Teorema Ergódico en cadenas de Markov (\cite{norris1998markov}). En beneficio del mencionado teorema, basta con construir una cadena de Markov de estados continuos a tiempos discretos recurrente e irreducible y aperiódica cuyo kernel de transición cumpla con balance detallado con respecto a $f(x)$. Para ahondar en detalles se requiere de un extenso estudio en cadenas de Markov a tiempo discreto con una cantidad numerable no finita de estados, donde en lugar de la usual matriz de transición se usa un operador de salto llamado kernel de transición. Un estudio completo del algoritmo de Metropolis-Hastings se encuentra en (\cite{mengersen1996rates}).

Pese a la enrevesada formalización del método, la estructura del algoritmo Metropolis-Hastings basta para dilucidar en ciertos detalles. Primeramente, del paso uno es claro que se debe elegir una distribución asociada a la variable propuesta $Y$ de forma que no sea un problema simular de esta. Además, dado que la propuesta se acepta o rechaza para ser realización de la variable $X$ asociada a la distribución objetivo, entonces se debe eligir una distribución de $Y$ tal que $supp\{X\} \subset supp\{Y\}$, que denota el soporte de la variable $Y$ debe contener el soporte de la variable objetivo $X$.

Asimismo, el segundo paso del algoritmo Metropolis-Hastings se observa la prescindencia del factor de normalización de la distribución objetivo puesto que solo es relevante el cociente $\frac{f(y)}{f(x)}$, un atributo peculiar de la distribución posterior. Paralelamente, el tercer paso del algoritmo se observa la propiedad de Markov, ya que la aceptación de la propuesta para pertenecer a la cadena depende estocásticamente del estado que lo precede unicamente.

\section{Forward map aproximado}

Hay que destacar que, una vez abordado el enfoque bayesiano para el problema inverso y el método MCMC para obtener la distribución posterior de los parámetros del modelo bajo estudio, el reto a superar recae en la implementación de la metodología descrita. Es preciso tener presente que pese a ser una metodología ampliamente funcional, no se cierra la puerta a inquerir mejoras en su procedimiento. Es por ello que la propuesta principal del trabajo de tesis es la introducción del forward map aproximado.

El forward map aproximado $F_{\theta}^{*}$ es una construcción multifacética basada en aproximaciones de estadística espacial. Tal aproximación pretende crear una distribución posterior aproximada $\tilde{\pi}(\theta|\mathbf{y})$ con la sustitución del forward map a su versión aproximada, de forma que la simulación de esta nueva distribución posterior sea más eficiente por el algoritmo Metropolis-Hastings. 

\subsection*{Construcción del forward map aproximado}
Existen diferentes maneras de hacer las aproximaciones del forward map. La forma propuesta es considerar una discretización del espacio de parámetros $\Theta$. Usando coordenadas canónicas, cada $\theta \in \Theta \subset \mathbb{R}^d$ se puede escribir como $\theta = (\theta_1, \theta_2, \cdots, \theta_d)$. Consideremos para cada coordenada un intervalo $[\theta_i^{min},\theta_i^{max}]$ para toda $i \in \{1,\cdots,d\}$, donde $\theta_i^{min}$ y $\theta_i^{max}$ son cotas para el espacio de parámetros que contenga la masa de probabilidad, puede pensarse en tomar cuantiles de $0.01$ y $0.99$ respectivamente de la distribución marginal a priori para $\theta_i$. Posteriormente, tomemos una partición equidistante en $M$ puntos para cada coordenada. Más llanamente, la partición de la coordenada $i-$ésima es el conjunto $\mathcal{M}_i =\{ \theta_i^{(1)}, \theta_i^{(2)},\cdots , \theta_i^{(M-1)}, \theta_i^{(M)}\}$ con $\theta_i^{(1)} = \theta_i^{min}$ y $\theta_i^{(M)} = \theta_i^{max}$. De esta forma, la partición crea una malla $\mathcal{M} = \mathcal{M}_1 \times \cdots \times \mathcal{M}_d$ del espacio parametral $\Theta$. Observese que la cardinalidad de $\mathcal{M}$ es de $M^d$. Denotemos por $\vartheta$ a los elementos de $\mathcal{M}$, así el enmallado se constituye de $M^d$ parámetros. Es decir, se establece que $\mathcal{M} = \{\vartheta_1, \vartheta_2, \cdots, \vartheta_{M^d}\}$, notese que $\vartheta_j \in \Theta$ para toda $j \in \{1,\cdots, M^d \}$.

Seguidamente, se requiere evaluar el forward map para cada elemento de $\mathcal{M}$, en la notación acordada se requiere $y_j = F(\vartheta_j)$, con $y_j = y_j(t)$ funciones continuas, que al estar restringidos a modelos dados por ODE es necesario un calculo intermedio. Dicho en otras palabras, solo se resuelven las ecuaciones diferenciales con los parámetros donde el enmallado interseca.

Para terminar, el forward map aproximado para parámetros $\theta \notin \mathcal{M}$ se propone buscar a los $k$ vectores de parámetros $\vartheta$ más cercanos en distancia euclideana a $\theta$, denotando estos $k$ vecinos por $\vartheta^{(1)}, \cdots, \vartheta^{(k)}$ cada uno a una distancia $d_1, \cdots, d_k$ de $\theta$, respectivamente; con $d_1 \leq d_2 \leq \dots \leq d_k$. La aproximación al forward map con $k$ vecinos en una malla con $M$ particiones es
\begin{align}
    \tilde{F}^{k}_M(\theta) = \sum_{j = 1}^{k} \omega_j F \left(\vartheta^{(j)}\right)
    \label{2.4.01}
\end{align}
donde $\omega_j = d_j^{-1}/ \sum_{i=1}^{k} d_i^{-1}$, que es una suma ponderada inversamente por las distancias, ya que se pretende que parámetros `lejanos' de $\theta$ tengan menor relevancia que los `cercanos'. 


\subsection*{Consistencia y utilidad del forward map aproximado}

La preeminencia potencial del uso del forward map aproximado para el estudio del problema inverso reside parcialmente en un menor tiempo de ejecución de la implementación en contraste con su análogo ordinario. Existen modelos de ecuaciones diferenciales cuya implementación del problema inverso bajo enfoque bayesiano toma días en ejecutarse, vease \cite{Anel}.
Debido a la reticente necesidad humana en aplicaciones impostergables es imperativo agilizar el tiempo en operación. Concisamente en modelos epidemiológicos la rapidez de acción se torna fundamental como lo fue en la predicción de ocupación hospitalaria por COVID-19 en la zona metropolitana de la CDMX (\cite{capistran2020forecasting}).

Por otro lado, fuera de las aplicaciones prácticas del método esbozado con el forward map aproximado, es de sumo interés matemático el estudio de la consistencia del método propuesto. Recordemos que tenemos espacio de maniobra para la construcción del forward map aproximado al mover la densidad de puntos en la malla $M$ así como en la cantidad de vecinos $k$ usados para la suma ponderada. De esta misma forma, tras sustituir el forward map aproximado (\ref{2.4.01}) en la expresión (\ref{2.2.04}) obtenemos la distribución posterior aproximada de los parámetros del modelo. Formalizando, salvo un factor de normalización definimos
\begin{align}
    \tilde{\pi}^{k}_M(\theta|\mathbf{y}) \propto \left(\frac{1}{2\pi \sigma^2}\right) ^{n/2}\exp \left \{  -\frac{1}{2\sigma^2}\sum_{i = 1}^{n} \left(y_i - F^k_M(\theta)\right)^2 \right \}\pi(\theta),
    \label{2.4.02}
\end{align}
la distribución posterior aproximada dado las observaciones $\mathbf{y} = (y_1,\cdots, y_n)$ al tiempo $t_1, \cdots, t_n$; respectivamente. Consideremos que la aproximación a la distribución posterior usando la metodología ordinaria del problema inverso es en efecto una aproximación plausible usando la \textit{distancia de Kullback-Leibler}.

Si $f(x)$ y $g(x)$ son funciones de densidad de probabilidad. La distancia de Kullback-Leibler entre $f$ y $g$ se define como
\begin{align*}
    D(f,g) = \int f(x)\log \left(\frac{f(x)}{g(x)}\right) dx.
\end{align*}
Se puede mostrar que $D(f,g) \geq 0$ y $D(f,f) = 0$ (\cite{wasserman2006all}).
Definamos la distancia entre las distribuciones posteriores como
\begin{align*}
    c^k_M = D \left( \pi(\theta|\mathbf{y}),\tilde{\pi}^{k}_M(\theta|\mathbf{y})\right)
\end{align*}

Diremos que el forward map aproximado es \textbf{consistente} si para una cantidad de vecinos $k$ fijo y conocido, las distancias 
\begin{align*}
    c^k_M \rightarrow 0 \:\:\:\:\:\: \text{a medida que}\:\:\:\:\:\: M \rightarrow \infty
\end{align*}

En el caso de un vecino ($k = 1$) la consistencia es trivial puesto que al hacer la malla más fina, el forward map aproximado será idéntico al forward map ordinario.

El estudio para $k > 1$ requiere de una estructura teórica sobre las cualidades particulares de la ecuación diferencial del modelo en cuestión. De forma que para indagar sobre la consistencia ees necesario usar técnicas heurísticas. En el capítulo siguiente se consideran modelos paramétricos particulares aplicando las metodologías del problema inverso con y sin aproximación para así cotejar empíricamente cierto grado de convergencia.









% Esta estructura de aproximación se presta a la interpretación de que la información obtenida en mallas finas sea transmitidas a los vecinos.

\chapter{Análisis Exploratorio}\label{chap:eda}

\section{Bases de datos}


\begin{figure}[H]
\centering
\includegraphics[width = 11cm]{ggmap_7_entidades_sample_02}  % using \graphicspath specification
\caption{Selección de 7 estados de la República Mexicana para la modelación. Este mapa fue construido con el uso de \texttt{ggmap} \citep{Kahle2013ggmap} y requirió la activación de los servicios de Google Maps.}
\label{fig:selecc_entidades_01}
\end{figure}

\chapter{Modelación espacio-temporal} \label{chap:modeling}

\section{Modelos espaciales} \label{sec:spatial_models}

\subsubsection{Modelo Poisson-Gamma} \label{subsubsec:bayesian_poisson_gamma}

Para este tipo de modelos es posible utilizar representaciones gráficas que reflejen la estructura de dependencias en la jerarquía


\chapter{Conclusiones y trabajo futuro}

Las aproximaciones de la distribución posterior dadas por las aproximaciones del forward map son en general útiles y simples de manejar. En cada uno de los tres modelos estudiados se llegó a que existe una aproximación decente con un menor tiempo de ejecución. Sin embargo, aunque es cierto que comparaciones uno a uno el método aproximado es mejor en tiempo de ejecución, no lo es en conjunto. Esto quiere decir, que la investigación de la mejor aproximación conlleva una mayor inversión temporal. 
Más aún, considérese que la experimentación se ha realizado con aproximaciones del forward map de la misma cantidad de vecinos cercanos y misma resolución de malla teniendo un grado de aproximación óptimo con tres vecinos y una definición de malla de 50  en cada uno de los modelos presentados. Por razonamiento inductivo, podemos conjeturar que tendremos una aproximación decente en cualquier otro modelo de EDO's con la misma regla de aproximación, esperando una mejora en el tiempo de ejecución previo a realizar la metodología con forward map ordinario. 

Cabe resaltar que las aproximaciones hechas con más vecinos fueron rechazadas por deficiencia de la convergencia experimental en la distribución posterior. Este caso se atribuye a que agregar muchos vecinos, como lo fue para aproximaciones de $8$ vecinos cercanos, añade a la aproximación de la distribución posterior un \textit{ruido} que perturba considerablemente la distribución posterior resultante.

Por otro lado, la incorporación de la regulación de las unidades tomó un papel fundamental para la metodología aproximada. En teoría hemos visto que la ausencia de esta resulta en aproximaciones sesgadas a aquellos parámetros cuyas unidades sean la de menor orden de magnitud. En la práctica, el uso de la metodología sin la regularización de las unidades produce aproximaciones burdas para los parámetros con mayor orden de magnitud. En el caso del modelo SIR, considerar la resolución del problema inverso sin la regularización de las unidades, ocasiona que la distribución posterior aproximada, solo aproxima marginalmente para $\beta$. Esto se debe a que sin la regularización de las unidades y debido a la construcción ponderada  inversamente a la distancia en el forward map aproximado, es que el orden de magnitud de las unidades de $\gamma$ al ser mayores que las mismas de $\beta$, son insignificantes para la aproximación ocasionando el sesgo mencionado. Veamos el caso de cinco vecinos cercanos en la Fig. \ref{Fig. Regulacion_1}
y el caso de ocho vecinos cercanos en la Fig. \ref{Fig. Regulacion_2} sin la regularización de las unidades.

\begin{figure}[H] 
    \centering 
    \includegraphics[width = 17 cm ]{img/Unidades_1.png} 
    % \caption{}
    % \label{Fig. }
\end{figure} 
\begin{figure}[H] 
    \centering 
    \includegraphics[width = 17 cm ]{img/Unidades_2} 
    \caption{Distribuciones marginales posteriores aproximadas (rojo) con forward map aproximado a cinco vecinos cercanos y una malla de resolución 10,15,30,50 de izquierda a derecha, sin regularización de las unidades. En azul la distribución posterior con forward map ordinario para modelo SIR.}
    \label{Fig. Regulacion_1}
\end{figure} 

\begin{figure}[H] 
    \centering 
    \includegraphics[width = 17 cm ]{img/Unidades_3.png} 
    % \caption{}
    % \label{Fig. }
\end{figure} 
\begin{figure}[H] 
    \centering 
    \includegraphics[width = 17 cm ]{img/Unidades_4} 
    \caption{Distribuciones marginales posteriores aproximadas (rojo) con forward map aproximado a ocho vecinos cercanos y una malla de resolución 10,15,30,50 de izquierda a derecha, sin regularización de las unidades. En azul la distribución posterior con forward map ordinario para modelo SIR.}
    \label{Fig. Regulacion_2}
\end{figure} 

Existen multiples formas de mejorar el algoritmo para la solución del problema inverso en modelos dados por EDO's. Una de ellas considerada a trabajo futuro es la creación de mallas no rectangulares, que además se vaya construyendo de forma adaptativa, esto quiere decir que se deje correr la cadena del algoritmo Metropolis-Hastings de forma ordinaria una cantidad fija de iteraciones y en cada paso se construya la malla. Esto con la idea de que los puntos de la discretización que son poco probables no se vean sobre-representados.

También considerado a futuro, es de interés trabajar con distintos modelos de EDO's a los presentados en este proyecto, pues de esta forma se podrían encontrar casos en los que la metodología aproximada propuesta aquí resulte con irregularidades, donde el tratamiento de estas conlleve a la mejora del algoritmo. Además de irregularidades como la que sucede en el modelo logístico, vea la Fig. \ref{Fig. Aprox log 3v}, donde se muestra un sesgo a la izquierda de la distribución posterior aproximada pese a que ya se regularizaron las unidades.

Por otro lado, es de interés implementar cuantitativamente el grado de aproximación que tiene las posteriores con su análogo ordinario. Proponiendo la implementación de las distancias de Kullback-Leibler como sustitución de la simple inspección de la aproximación.

% Bibliography
\bibliographystyle{apacite}  % APA style according to apacite package: see mypreamble.sty line 21
\bibliography{references}

% Appendix
\appendix
\include{chapters/appendix}

\backmatter

\end{document}

% LaTeX references (technical issues, workarounds)

% https://www.overleaf.com/learn/latex/Management_in_a_large_project ('including files' section)
% https://tex.stackexchange.com/questions/64839/how-to-change-listing-caption
% https://stackoverflow.com/questions/2709898/change-list-of-listings-text
% https://tex.stackexchange.com/questions/664/why-should-i-use-usepackaget1fontenc