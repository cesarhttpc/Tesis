\chapter{Introducción}

En una amplia gama de disciplinas se utilizan modelos que pretenden describir diversos fenómenos. La inmensa mayoría de estos modelos son matemáticos y establecen relaciones entre variables expresadas mediante ecuaciones dependientes de ciertos parámetros. El tipo de modelos consideraros son aquellos cuyas variables se pueden expresar ya sea en una forma funcional o en forma de ecuación diferencial ordinaria (ODE).

\section*{Proceso de modelación}

El proceso de modelación de un fenómeno involucra tres sectores principales.Al investigar un fenómeno, se identifican, proponen y prueban las variables que influyen en él. Este proceso, conocido como \textbf{parametrización}, implica seleccionar las variables relevantes para la descripción del modelo.

Una vez determinadas las variables del modelo, se establecen explícita o implícitamente las relaciones entre ellas. Se pueden proponer modelos lineales generalizados, series temporales, ecuaciones diferenciales o simplemente ecuaciones algebraicas con las variables y parámetros de ajuste, que en adelante llamaremos simplemente parámetros. Este sector del proceso se conoce como \textbf{modelación directa}.

El último sector de la modelación es el \textbf{problema inverso}. Como su nombre indica, se refiere a inferir los parámetros del modelo a partir de una muestra del fenómeno modelado. Esto es opuesto al problema directo, que busca describir explícitamente la dinámica del fenómeno a partir de parámetros fijos \cite{calvetti2018inverse}. 

El proceso de modelación no sigue una línea recta. Es por ello que se optó por llamar a sus partes como sectores de modelación en lugar de etapas de modelación, ya que todos estos sectores siguen un camino cíclico según las virtudes de cada modelo en particular. 

Una vez aclarado que los modelos considerados son aquellos que pueden expresarse como una ODE dependientes de parámetros, la tesis se centra en el problema inverso desde el enfoque bayesiano. En el capítulo 2 se describe el paradigma bayesiano aplicado al problema inverso. Además de los antecedentes necesarios para hacer de este texto autocontenido. 

En el capítulo 3 se describe las implementaciones realizadas para la inferencia bayesiana de los parámetros de tres modelos particulares según el paradigma convencional. Además, se propone un método aproximado para realizar el mismo procedimiento, cuestionando la factibilidad de esta propuesta como sustituto del paradigma original.


















% \newpage







% \textcolor{blue}{Explicar a grandes rasgos el problema inverso. Poner sus inconvenientes y dificultades así como escribir porque es de interés. }

% \vspace{2 cm}

% \textcolor{blue}{
% Luego, hablar de la inferencia bayesiana como método para hacer inferencia sobre el problema inverso, explicar la teoría de la estadística bayesiana así como sus paradigmas generales.}

% \vspace{2 cm}

% \textcolor{blue}{
% Hablar de MCMC como método de simulación para la distribución posterior de la estadística bayesiana. Explicar la teoría a grandes rasgos, esbozar el algoritmo.}





