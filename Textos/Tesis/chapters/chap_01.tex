\chapter{Introducción}

El estudio de la violencia en México no es un tema nuevo, sino de antaño para la sociedad, así como la preocupación por resolverlo. En consecuencia, a través del tiempo se han planteado diversas explicaciones, tanto teóricas como empíricas para abordar el fenómeno e identificar factores que llevan a los grupos humanos a ejercer la violencia en los territorios. Estas explicaciones han sido abordadas desde diversas visiones como  las geográficas, políticas, económicas, y sociológicas, por nombrar las más estudiadas. Ante una realidad tan compleja como es el estudio de la violencia, observamos la necesidad de abordarlo desde las ciencias estadísticas y computacionales, y es desde esta necesidad que desarrollamos esta tesina \ldots


\section{Problema} \label{sec:problem}

México actualmente atr \cite{Kahle2013ggmap}




\section{Contribución} \label{sec:contribution}

En literatura
\section{Organización de la tesis}






