\chapter{Introducción}

En una amplia gama de disciplinas se utilizan modelos que pretenden describir algún fenómeno. La inmensa mayoría de estos son modelos matemáticos, teniendo relaciones entre variables expresadas por ecuaciones dependientes de ciertos parámetros. El tipo de modelos consideraros son aquellos cuyas variables se pueden expresar ya sea en una forma funcional o en forma de ecuación diferencial ordinaria (ODE).

\section{Proceso de modelación}

El proceso de modelación para cierto fenómeno involucra una estructura cohesionada en tres principales sectores. Al investigar un fenómeno se plantea, propone y prueban las variables de las cuales el fenómeno responde a cambios. Dicho de otra forma, se seleccionan las variables que influyen en la descripción del modelo. Dicho sector se conoce como \textbf{parametrización}. 

Posteriormente, una vez postulado las variables del modelo, sigue estructurar ya se explícita o implícitamente las relaciones entre las mencionadas variables. Se puede proponer modelos lineales generalizados, series temporales, ecuaciones diferenciales o simplemente una ecuación algebraica con las variables involucradas y además parámetros de ajuste que en adelante conoceremos solamente como parámetros. Este sector de la modelación es conocido como \textbf{modelación directa}. 

El último gran sector de la modelación es \textbf{el problema inverso}. Como su nombre sugiere, se refiere al problema de abordar cierta inferencia sobre los parámetros del modelo dada una muestra propia del fenómeno modelado. Que es lo contrario al problema directo, el cual dado unos parámetros fijos busca encontrar explícitamente la dinámica del fenómeno. 

El proceso de modelación no sigue una línea recta. Es por ello que se optó por llamar a sus partes como sectores de modelación en lugar de etapas de modelación, ya que todos estos sectores siguen un camino cíclico según las virtudes de cada modelo en particular. 

Una vez aclarado que los modelos considerados son aquellos que pueden expresarse como una ODE dependientes de parámetros, la tesis se centra en el problema inverso desde el enfoque bayesiano. En el capítulo 2 se describe el paradigma bayesiano aplicado al problema inverso. Además de los antecedentes necesarios para hacer de este texto autocontenido. 

En el capítulo 3 se describe las implementaciones realizadas para la inferencia bayesiana de los parámetros de tres modelos particulares según el paradigma convencional. Además, se propone una forma aproximada para realizar el mismo procedimiento, donde se cuestiona la factibilidad de la propuesta como sustituto del paradigma original.


















% \newpage







% \textcolor{blue}{Explicar a grandes rasgos el problema inverso. Poner sus inconvenientes y dificultades así como escribir porque es de interés. }

% \vspace{2 cm}

% \textcolor{blue}{
% Luego, hablar de la inferencia bayesiana como método para hacer inferencia sobre el problema inverso, explicar la teoría de la estadística bayesiana así como sus paradigmas generales.}

% \vspace{2 cm}

% \textcolor{blue}{
% Hablar de MCMC como método de simulación para la distribución posterior de la estadística bayesiana. Explicar la teoría a grandes rasgos, esbozar el algoritmo.}





