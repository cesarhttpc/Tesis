\chapter{Conclusiones y trabajo futuro}

Las aproximaciones de la distribución posterior dadas por las aproximaciones del forward map son en general útiles y simples de manejar. En cada uno de los tres modelos estudiados se llego a que existe una aproximación decente con un menor tiempo de ejecución. Sin embargo, aunque es cierto que comparaciones uno a uno el método aproximado es mejor en tiempo de ejecución, debe de considerarse también el tiempo en conjunto necesario para conocer la mejor aproximación. Puesto que se realizaron los experimentos del forward map con la misma cantidad de vecinos y número de puntos en enmallado, y además encontramos que en cada uno de ellos se obtuvo una aproximación buena con la malla de tamaño 50x50 con 3 vecinos, suponemos que esto sucede en una amplia gamma de modelos. Así, al analizar un nuevo modelo, tomamos simplemente la aproximación mencionada esperando una mejora sustancial del tiempo de ejecución.

Existen varias maneras de implementar un mejor algoritmo, una de ellas considerada a futuro es la creación de mallas no rectangulares, que además se vaya construyendo de forma adaptativa, esto quiere decir que se deje correr la cadena del algoritmo Metropolis-Hastings de forma ordinaria una cantidad fija de iteraciones y en cada paso se construya la malla. Esto con la idea de que los puntos de la discretización que son poco probables no se vean sobre-representados.