
\chapter{Antecedentes}


\section{Solución bayesiana a  problemas Inversos}

\textcolor{blue}{Tratar a fondo sobre la teoría de problemas inversos. Cómo se define. Que es el Forward map. Métodos convencionales (superficialmente para resolución)}





% \subsection{Método 1}
\textcolor{blue}{
Método para problemas inversos (lineales)}

\subsubsection{Método por estadística bayesiana}
\textcolor{blue}{Explicar como abordar el problema inverso con estadística bayesiana}


\section{Forward map aproximado y consistencia}

\subsubsection{Teoría}
\textcolor{blue}{Aquí explicar la teoría clásica para la estadística bayesiana.}

\subsubsection{Ejemplos}


\section{MCMC}

\textcolor{blue}{Introducción a la simulación por Monte Carlo.}

% \subsection{Método 1: (Importance of sampling: apendice)}

\subsection{Metropolis-Hastings}:

\textcolor{blue}{Explicar teoría de MCMC, teoremas de cadenas de Markov(apendice)}


\section{Necesidad de trabajar con aproximaciones del Forward map}

\vspace{1 cm} 

\textcolor{blue}{Esbozar el algoritmo y explicar la practicidad de este. Poner un ejemplo ilustrativo.}

